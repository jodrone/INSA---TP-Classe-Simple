\documentclass[a4paper,french,10pt,twoside]{article}
\usepackage[utf8]{inputenc}
\usepackage[T1]{fontenc}
\usepackage[english, francais]{babel}
\usepackage{listings}
\usepackage{amsmath}
\usepackage{graphicx}
\usepackage[colorlinks=true,linkcolor=black,citecolor=blue,urlcolor=blue]{hyperref}
\usepackage{fancyhdr}
\usepackage{fullpage}

\title{TP 3IF - C++ Classe Simple : Spécifications}
\author{T. Pourcelot \& J. Vincent}

\begin{document}

%%%%%%%%%%%%% Definition des en têtes et pieds de page

\pagestyle{fancy}
\fancyhf{} % on efface tout et on recommence
	% PIED DE PAGE :
	\fancyfoot[LO,RE]{3IF - Binôme 3213 \hfill{} \textbf{Sp\'ecifications - TP C++ - Classe Simple} \hfill{} \thepage }
	% Epaisseur des traits
	\renewcommand{\headrulewidth}{0pt}
	\renewcommand{\footrulewidth}{2pt}



%%%%%%%%%%%%%%%%%%
% Page de Garde  %
%%%%%%%%%%%%%%%%%%

\thispagestyle{empty}

\begin{minipage}[c]{0.5\linewidth}
  \vspace{50pt}
  \includegraphics[bb=0 50 350 100, scale=0.3]{images/insa.JPG}
\end{minipage}
\begin{minipage}[r]{0.5\linewidth}
  \vspace*{25pt}
  \begin{flushright}
	\LARGE D\'epartement \\ Informatique \\ \rule{\linewidth}{2pt}
  \end{flushright}
\end{minipage}

\vspace*{5cm}

\setlength\fboxrule{2pt}
\fbox{
	\begin{minipage}{0.95\textwidth}
    \begin{center}
		\vspace*{0.21cm}\Huge \color{blue} \textbf{\bsc{Document de Sp\'ecification \\et de conception} }
	\end{center}
	\end{minipage}
}

\vspace{1cm}

\begin{center}
  \Large \textbf{\bsc{TP-AC \\ Classe Simple}}\\
\end{center}

\begin{center}
	\vspace*{3cm}
	%Changer les noms ici
	\LARGE \bsc{Binôme 3213}\\ \vspace{10pt}\large \color{black} Tristan \bsc{Pourcelot} \\ Jordan \bsc{Vincent} \\
				   \vspace*{1.5cm}
				   %\Large Responsable : P. \bsc{Girard}\\

				   \vspace*{1cm}

				   \Large 3IF - Groupe 2 \hfill Ann\'ee scolaire 2011-2012\\
				   \vspace*{1cm}
				   \textbf{\Large Institut National des Sciences Appliqu\'{e}es de Lyon} \pagebreak
\end{center}

\section{Sp\'ecifications et d\'efinitions globales de la classe}

Notre classe a pour but de permettre de g\'erer un ensemble d'intervalles.\\
Un intervalle peut se d\'efinir par ses deux bornes, soit $borne_{inf}$ et $borne_{sup}$ qui sont deux entiers sign\'es. Les intervalles infinis ne sont pas pris en compte.\\ Autre cas particulier, les intervalles vides (tels que $borne_{inf}=borne_{sup}$) sont pris en compte.\newline{}

La classe gère le cas particuliers des ensembles d'intervalles disjoints. Deux intervalles disjoints sont deux intervalles qui n'ont aucun point en commun. \\

L'ensemble des intervalles est tri\'e, afin de faciliter les op\'erations de recherche et de calcul.\\

D'un point de vue pratique, cet ensemble est repr\'esent\'e par une liste chaîn\'e de structures du type Interval.


\section{Sp\'cifications des m\'ethodes}

\subsection{Constructeur de recopie :  M\'ethode IntervalSet(IntervalSet)}

Cette m\'ethode prend en argument un pointeur vers un objet de type IntervalSet d\'ejà construit. Il alloue l'espace m\'emoire n\'ec\'essaire pour stocker le contenu de l'IntervalSet pass\'e en argument.\\
CONTRAT : l'objet pass\'e en argument doit être un objet IntervalSet cr\'e\'e et correctement constitu\'e. Toutefois, cet objet peut être vide.

\subsection{Ajout d'un intervalle : m\'ethode AddInterval(Interval) }
Cette fonction prend en argument une structure de type \verb!Interval! et retourne un bool\'en t\'emoin du succès de l'op\'eration. Il convient à l'utilisateur de v\'erifier si l'ajout a \'et\'e correctement effectu\'e.\\
La m\'ethode v\'erifie la validit\'e de l'intervalle (disjonction avec la collection d\'ejà existante, puis, si ce test est correct, l'intervalle est ajout\'e à l'ensemble et la fonction retourne TRUE. Si ce test n'est pas concluant, l'intervalle est rejet\'e et la fonction retourne FALSE.

\subsection{Calcul de la r\'eunion de deux ensembles d'intervalles : M\'ethode Union(IntervalSet)}

\subsection{Intersection de deux ensembles d'intervalles : M\'ethode Intersection(IntervalSet)}

\section{Tests Unitaires et fonctionnels}

\subsection{Test fonctionnel N$^o$ 1 : Fonctionnement normal de la classe}
Entr\'ees <-> Sorties attendues

\subsection{Test fonctionnel N$^o$ 2 : Fonctionnement aux limites}
Entr\'ees <-> Sorties attendues

\subsection{Test Unitaire No 1 : M\'ethode Count}
idem

\subsection{Test Unitaire No 2 : M\'ethode Union}
pareil

\end{document}
