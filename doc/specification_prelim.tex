\documentclass[a4paper,french,10pt,twoside]{article}
\usepackage[utf8]{inputenc}
\usepackage[T1]{fontenc}
\usepackage[english, francais]{babel}
\usepackage{listings}
\usepackage{amsmath}
\usepackage{graphicx}
\usepackage[colorlinks=true,linkcolor=black,citecolor=blue,urlcolor=blue]{hyperref}
\usepackage{fancyhdr}
\usepackage{fullpage}

\title{TP 3IF - C++ Classe Simple : Spécifications}
\author{T. Pourcelot \& J. Vincent}

\begin{document}

%%%%%%%%%%%%% Definition des en têtes et pieds de page

\pagestyle{fancy}
\fancyhf{} % on efface tout et on recommence
	% PIED DE PAGE :
	\fancyfoot[LO,RE]{3IF - Binôme 3213 \hfill{} \textbf{Sp\'ecifications - TP C++ - Classe Simple} \hfill{} \thepage }
	% Epaisseur des traits
	\renewcommand{\headrulewidth}{0pt}
	\renewcommand{\footrulewidth}{2pt}



%%%%%%%%%%%%%%%%%%
% Page de Garde  %
%%%%%%%%%%%%%%%%%%


\begin{minipage}[c]{0.5\linewidth}
  \vspace{50pt}
  \includegraphics[bb=0 50 350 100, scale=0.3]{images/insa.JPG}
\end{minipage}
\begin{minipage}[r]{0.5\linewidth}
  \vspace*{25pt}
  \begin{flushright}
	\LARGE D\'epartement \\ Informatique \\ \rule{\linewidth}{2pt}
  \end{flushright}
\end{minipage}

\vspace*{1cm}
\thispagestyle{empty}
\begin{center}
  \Huge \textbf{\bsc{TP C++ \\ Classe Simple}}\\
\end{center}
\vspace*{1.3cm}

\setlength\fboxrule{2pt}
\fbox{
	\begin{minipage}{0.95\textwidth}
    \begin{center}
		\vspace*{0.21cm}\LARGE \color{blue} \textbf{\bsc{Dossier de Sp\'ecification} }
	\end{center}
	\end{minipage}
}


\begin{center}
	\vspace*{1cm}
	%Changer les noms ici
	\LARGE \bsc{Binôme 3213}\\ \vspace{10pt}\large \color{black} Tristan \bsc{Pourcelot} \\ Jordan \bsc{Vincent} \\
				   \vspace*{1.5cm}
				   %\Large Responsable : P. \bsc{Girard}\\

				   \vspace*{1cm}

				   \Large 3IF - Groupe 2 \hfill Ann\'ee scolaire 2011-2012\\
				   \vspace*{1cm}
				   \textbf{\Large Institut National des Sciences Appliqu\'{e}es de Lyon} \pagebreak
				 \end{center}





%%%%%%%%%%%%%%%%%%
%  Corps du doc	 %
%%%%%%%%%%%%%%%%%%

\part{Spécifications générales du programme}
La classe IntervalSet a pour rôle principal de gérer un ensemble d'intervalles

\part{Définitions Globales}

\begin{enumerate}
\item{définition d'un intervalle}
\newline{}
Un intervalle est un ensemble défini par ses deux bornes borne\_inf et borne\_sup.
Ces deux bornes sont de type \verb!long signed int!.\\
D'autre part on a bien évidemment $borne_sup \geq borne_inf$\\
(ie -> on accepte un intervalle composé d'un seul élément t.q. borne\_sup = borne\_inf)\\
De même, il faut noter qu'il est impossible de traiter un intervalle infini.


\item{Définition d'un ensemble d'intervalle}
\newline{}
\end{enumerate}
Un ensemble d'intervalles est un groupe d'intervalles disjoints, çad qu'ils n'ont aucun élément en commun.\\
Ainsi, deux intervalles [borne\_inf1 , borne\_sup1] et [borne\_inf2 , borne\_sup2] t.q. borne\_sup1 = borne\_inf2 ne sont pas disjoints!
\part{Rôle de la classe}

La classe IntervalSet a pour rôle principal de gérer un ensemble d'intervalles
On peut ajouter un intervalle ou un ensemble d'intervalles à la classe.
D'autre part, cet ensemble d'intervalles est trié dans l'ordre croissant des valeurs inférieures.
La classe permet d' afficher sur la sortie standard la liste des intervalles qu'il contient et leur nombre.
A chaque intervalle de la classe corespond un indice. Cet indice permet de repérer de manière unique un intervalle de l'ensemble.
Cet indice permet de sélectionner ou de supprimer un intervalle de l'ensemble. 
On peut effectuer des opérations ensemblistes entre deux sets d'intervalles (union et intersection)

\begin{itemize}
	\item{\large Caractéristiques d'un objet de notre classe }
	\begin{itemize}
		\item{Notre classe contient une collection triée de structures \verb!Interval! contenant l'ensemble des intervalles}
		\item{Un nombre d'index \verb!interval_count! qui répertorie la longueur de la liste d'intervalles}
		\item{FACULTATIF : Un attribut \verb!etendue! de type \verb!Interval! qui recense l'étendue de la collection d'intervalle}
	\end{itemize}
	\item{\large Spécifications de l'interface publique}
	\begin{itemize}
		\item{Constructeur \verb!IntervalSet()!}\\
			Ce constructeur ne prends aucun argument, et ne fait qu'initialiser l'espace mémoire pour la structure\\
		\item{Constructeur de recopie \verb!IntervalSet(Bordel...)!  }\\
			Ce constructeur prends en argument un pointeur vers un IntervalSet existant, et recopie 
			la structure pointée dans un nouvel espace qu'il alloue.\\
		\item{Procédure d'affichage \verb!Display!}\\
			Cette procédure ne prends pas d'argument. Elle affiche les différents intervalles de la collection, en partant du premier
			(borne\_inf la plus petite), au plus grand (borne\_sup la plus grande).\\
			La sortie attendue est :\\ 
				\verb!"L'intervalle numéro 0 commence à borne_inf_O et se termine à borne_sup_0."!
				\dots \\
				\verb!"L'intervalle numéro N commence à borne_inf_n et se termine à borne_sup_n."!\\
		\item{Fonction booléen : \verb!AddInterval(Interval)!}\\
			Cette méthode prends en argument un intervalle de type \verb!Interval! et renvoie un booléen
			signifiant le succès ou non de l'ajout de l'intervalle à la collection.
			Elle teste la validité de cet intervalle (disjonction avec la collection déjà existante)
			puis, si ce test est correct, l'ajoute à l'ensemble à la bonne position (respect de l'ordre).
			Si jamais l'ensemble n'est pas disjoint de la collection, il n'est pas ajouté à l'ensemble et 
			la méthode retourne la valeur FALSE.
			Contrat : l'argument Interval doit être défini de manière correcte (voir doc du type Interval)
		\item{Fonction booléen : \verb!AddIntervalSet(IntervalSet)!}\\
			Cette méthode prends en argument un ensemble d'intervalles de type \verb!IntervalSet! et renvoie un booléen
			signifiant le succès ou non de l'ajout de la collection d'intervalles.
			Elle teste la validité de chaque intervalle (disjonction avec la collection déjà existante)
			puis, si tout les tests sont corrects, on ajoute un à un les différents intervalles contenus dans l'ensemble
			passé en argument.
			Si jamais au moins un intervalle n'est pas disjoint de la collection, aucun intervalle n'est ajouté, et la 
			méthode retourne la valeur FALSE.
		\item{Fonction Entier : \verb!Count()!}\\
			Cette méthode renvoie le nombre d'intervalles contenus dans l'ensemble, et 0 si l'ensemble est vide.
		\item{Fonction Interval : \verb!GetInterval( Entier indice )!}\\
			Cette méthode renvoie une structure du type \verb!Interval! dont l'indice dans la structure est passé en
			argument.\\
			CONTRAT : Cet indice doit être compris entre $0$ et IntervalSet.Count().\\
			Si la collection de IntervalSet est vide (IntervalSet.Count=0), cette méthode renvoie l'ensemble [0,0]
		\item{Fonction Booléen : \verb!Remove(Entier indice)!}\\
			Cette méthode prends en argument un entier \verb!indice! et supprime l'intervalle dont l'index dans la collection correspond
			à l'indice passé en argument.\\
			CONTRAT : Cet indice doit être compris entre $0$ et IntervalSet.Count().\\
			En cas de succès, cette méthode renvoie la valeur \verb!TRUE!.\\
			En cas d'échec (indice $>$ IntervalSet.count, erreur lors de la suppression\dots), cette méthode renvoie FALSE. 
		\item{Fonction IntervalSet :\verb! Union(IntervalSet a_intervalle)!}\\
			Cette méthode renvoie la réunion d'intervalle de deux ensembles.\\
			Elle prend comme argument une collection d'intervalles de type IntervalSet.
			On désigne S la collection d'intervalle de sortie, A la collection passée en argument, et C la collection propre de l'objet.\\
			On teste les différents intervalles de A et de C, afin de trouver les points communs (jointures), et si des intervalles se recoupent,
			on les fusionnent. On ajoute ensuite cette collection d'intervalles fusionnés et disjoints à la collection S, avant de la retourner.
		\item{Fonction IntervalSet : \verb! Intersection(IntervalSet a_intervalle!}\\
			Cette méthode renvoie l'intersection entre deux ensembles d'intervalles.\\
			Elle prend comme argument une collection d'intervalles de type \verb!IntervalSet!.
			On désigne S la collection d'intervalle de sortie, A la collection passée en argument, et C la collection propre de l'objet.\\
			On teste les différents intervalles de A et de C, afin de trouver les points communs (jointures), et si des intervalles se recoupent,
			on ne sélectionne que l'intersection. On ajoute ensuite cette collection d'intervalles à la collection S, avant de la retourner.
	\end{itemize}
\end{itemize}

\end{document}
